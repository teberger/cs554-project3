\documentclass{article}
\usepackage{minted}
\usepackage{hyperref}
\usepackage{amsmath}
\usepackage[left=1in, right=1in]{geometry}

\begin{document}
\title{Project 2 Report}
\author{Taylor Berger, Zachary Friedland, Jianyu Yang}
\maketitle

\section{Language Decisions}
\paragraph{} Our group continued to use Python for the
ease of expressing high level concepts and removal of memory
management from the project. We also used Git for version control the
collaboration tools. The repository can be found at
\href{https://www.github.com/teberger/cs554-project3}{Taylor's
  Github}.

\paragraph{} We elected to use Zach's homework since he integrated the
support of ANTLR and its powerful parsing constructs to make
constructing the AST an easy task. We also took advantage of PyGraphViz for the AST and CFG visualization tasks. 

\section{Scanning, parsing and displaying the AST}
Scanning, parsing and constructing the AST was done using \href{https://pypi.python.org/pypi/antlr_python_runtime/3.1.3}{ANTLR} and the grammar found in the \verb{simple.g{ file displayed below.

\begin{minted}{antlr}
grammar simple;

options{
	language=Python;
	output=AST;
	ASTLabelType=CommonTree;
	backtrack=true;
}
/* Arithmetic operators, in order of precedence. */
MULT	:	'*';
PLUS	:	'+';
MINUS	:	'-';

/* Boolean operators, in order of precedence. */
NOT	:	'not';
AND	:	'&';
OR	:	'|';
RELOP	:	('=' | '<' | '<=' | '>' | '>=');

/* If / else. */
IF	:	'if';
THEN	:	'then';
ELSE	:	'else';
ENDIF	:	'fi';
SKIP	:	'skip';

/* Do while. */
WHILE	:	'while';
DO	:	'do';
ENDWHILE	:	'od';

/* Misc. */
GETS	:	':=';
SEMI	:	';';
LPAREN	:	'(';
RPAREN	:	')';
BLOCK	:	'block';
UNARY	:	'unary';

/* Atoms. */
BOOLEAN	:	('true' | 'false');
IDENT	:	('a'..'z' | 'A'..'Z')('a'..'z' | 'A'..'Z' | '0'..'9')*;
INTEGER	:	('0'..'9')+;

/* Ignore whitespace. */
WS	:	(' ' | '\t' | '\n' | '\r' | '\f')+ {$channel = HIDDEN;};


program
 	:	block
 	;

block
	:	statement+
		-> ^(BLOCK statement+)
	;

/* Arithmetic expressions - craziness due to precendence! */
arith_expr
	:	add_expr
	;
add_expr
	:	sub_expr (PLUS^ sub_expr)*
	;
sub_expr 
	:	mult_expr (MINUS^ mult_expr)*
	;
mult_expr 
	:	unary_expr (MULT^ unary_expr)*
	;
unary_expr
	:	MINUS arith_atom
		-> ^(UNARY MINUS arith_atom)
	|	PLUS arith_atom
		-> ^(UNARY PLUS arith_atom)
	|	arith_atom
	;
arith_atom
	:	(IDENT | INTEGER)
	|	LPAREN! arith_expr RPAREN!
	;

/* Boolean expressions - craziness due to precedence! */
bool_expr 
	:	or_expr
	;
or_expr 
 	:	and_expr (AND^ and_expr)*
 	;
and_expr
 	:	bool_atom (OR^ bool_atom)*
 	;
bool_atom
 	:	BOOLEAN
	|	NOT^ bool_atom
	|	LPAREN! bool_expr RPAREN!
	|	arith_expr RELOP^ arith_expr
	;

statement
    :   IDENT GETS^ arith_expr SEMI!
    |   SKIP SEMI!
    |   IF^ bool_expr THEN! block ELSE! block ENDIF! SEMI!
    |   WHILE^ bool_expr DO! block ENDWHILE! SEMI!
    ;
\end{minted}

Since the code to parse and generate the AST is auto-generated code, it will be omitted for now, but can be found in the code appendix at the end of the report. 


\section{Decorated AST}

\section{Conversion to a CFG}
Since displaying the graph was done through pygraphviz, we chose to
directly convert the AST given to us by ANTLR directly into an AGraph
object in pygraphviz. The following snippet expects an AST previously
parsed by ANTLR and converts it into an AGraph that represents the
CFG.

\begin{minted}{python}
def convert_ast_to_cfg(ast):
    graph = AGraph()
    id = 0
    program = ""
    revers_stack = []
    for x in ast.children:
        revers_stack.append(x)
    stack = []
    while(len(revers_stack) != 0):
        stack.append(revers_stack.pop())
    layer_stack = []
    last_id = 0
    un_solved_stack = []
    while(len(stack) != 0):
        x = stack.pop()
	print x.text
	print un_solved_stack
        if x.text == ":=" or x.text == "skip":
            if x.text == "skip":
                program = "skip";
            else:
                program = str(x.children[0].text) + " := " + render_short(x.children[1])
            graph.add_node(id, l = program)
            if(last_id != 0):
	        graph.add_edge(last_id, id)
            last_id = id
        elif x.text == "if":
            program = "if " + render_short(x.children[0])
            graph.add_node(id, l = program)
            if(last_id != 0):
		graph.add_edge(last_id, id)
            fi_node = copy(x.children[0])
	    else_node = copy(x.children[2])
            then_node = copy(x.children[1])
	    fi_node.token.text = "fi"
            else_node.token.text = "else"
            then_node.token.text = "then"
	    stack.append(fi_node)
            stack.append(else_node)
            stack.append(then_node)
            layer_stack.append(("if", id))
            last_id = id
        elif x.text == "then":
	    un_solved_stack.append("then")
            revers_stack = []
            for xx in x.children:
                revers_stack.append(xx)
            while(len(revers_stack) != 0):
                stack.append(revers_stack.pop())
        elif x.text == "else":
	    un_solved_stack.pop()
            un_solved_stack.append(last_id)
	    un_solved_stack.append("else")
            revers_stack = []
            for xx in x.children:
                revers_stack.append(xx)
            while(len(revers_stack) != 0):
                stack.append(revers_stack.pop())
            (statement, last_id) = layer_stack.pop()
	    layer_stack.append((statement,last_id))
        elif x.text == "fi":
	    un_solved_stack.pop()
            un_solved_stack.append(last_id)
	    layer_stack.pop()
	    last_id = 0
        elif x.text == "while":
            program = "while " + render_short(x.children[0])
            graph.add_node(id, l = program)
            if(last_id != 0):
		graph.add_edge(last_id,id)
	    od_node = copy(x.children[0])
            do_node = copy(x.children[1])
	    od_node.token.text = "od"
            do_node.token.text = "do"
	    stack.append(od_node)
            stack.append(do_node)
	    layer_stack.append(("while", id))
	    last_id = id
        elif x.text == "do":
            revers_stack = []
            for xx in x.children:
                revers_stack.append(xx)
            while(len(revers_stack) != 0):
                stack.append(revers_stack.pop())
        elif x.text == "od":
	    end_while_id = last_id
	    (statement, last_id) = layer_stack.pop()
            graph.add_edge(end_while_id, last_id)
	print "id:",id," statement: ",program
	if(x.text != "fi" and x.text != "od"):
	    s = ""
	    if(len(un_solved_stack) != 0):
	        s = un_solved_stack.pop()
	    while(s != "then" and s != "else" and s != ""):
	        graph.add_edge(s, id)
    		s = ""
	        if(len(un_solved_stack) != 0):
		    s = un_solved_stack.pop()
	    if(s == "then" or s == "else"):
	        un_solved_stack.append(s)
        id += 1
        program = ""
    return graph
\end{minted}

\section{Data Structures for Reaching Definitions}
Since python allows for some nice set notation, we decided to
implement everything using python's built in sets. Each set is
populated with tuples of ($x$, $l$) where $x \in$ variables(program)
and $l \in$ labels(program). 

\subsection{Data Flow Equations}
In each of our CFG's we annotate each node with a list of the variables being updated at that particular label. Since the grammar is relatively simple, there is only one assignment that can occur at the \emph{statement} level. To generate the Reaching Definitions for a given program, we first calculate the genSets and killSets for any given statement as:

\begin{minted}{python}
def getGenSets(cfg):

    nodes = cfg.nodes()
    labels = (node.name for node in nodes)
    variables = (node.attr["var"] for node in nodes)

    genSets = dict()
    for node, label, variable in zip(nodes, labels, variables):
        genSets[label] = {(variable, label)} if variable != '' else set()

    return genSets


def getKillSets(genSets):
    killSets = dict()

    for label, genSet in genSets.iteritems():
        killSets[label] = set()

        if genSet:
            for generation in genSet:
                variable = generation[0]

                for gs in genSets.values():
                    for v, l in gs:
                        if v == variable and l != label:
                            killSets[label] |= gs

    return killSets
\end{minted}

After creating the kill and gen sets for an arbitrary statement, we can construct the reaching definitions with the following bit of code:

\begin{minted}{python}
def getReachingDefinitions(cfg, startLabel):
    """
    Calculates the reaching definitions of a control-flow graph.

    :param AGraph cfg: The control-flow graph for which to calculate RDs.
    :param str startLabel: The label of the start node of the CFG.
    :rtype: dict[str, set[(str, str)]]
    """

    # Gen / Kill sets can be generated completely immediately.
    # IN / OUT start out empty initially.
    # The previousOut set is used to track changes to OUT sets during iteration.
    genSets = getGenSets(cfg)
    killSets = getKillSets(genSets)
    inSets = {label: set() for label in genSets}
    outSets = inSets.copy()
    previousOutSets = {label: None for label in genSets}

    # The iteration queue stores the next nodes to evaluate in the iteration.
    queue = deque([startLabel])

    while queue:
        label = queue.popleft()

        # Calculate (new) IN set.
        inSets[label] = getInSet(label, cfg, outSets)

        # Calculate the (new) OUT set, and check if it is different from the
        # previous version so we know whether or not to add the node's
        # successors to the iteration queue. The previous set is initially
        # populated by None, so every node will be visited at least once.
        outSet = getOutSet(label, genSets, killSets, inSets)
        if outSet != previousOutSets[label]:
            outSets[label] = outSet
            previousOutSets[label] = outSet
            queue.extend((s.name for s in cfg.successors(label) if s not in queue))

    return inSets
\end{minted}

The function to calculate the reaching definitions for a particular entry is $\text{RD(block)} = \text{RD(block)} - \text{killSet(block)} \cup \text{genSet(block)}$. This is applied iteratively until there are no changes in the previously calculated RD for any given block. There are two utility functions (\emph{getInSet} and \emph{getOutSet}) that are not mentioned here, but can be found in the code appendix. 

\section*{Code Appendix}
\subsection{AST To CFG Code}
\inputminted{python}{../parser.py}
\inputminted{python}{../parser2.py}

\subsection{Reaching Definitions}
\inputminted{python}{../reaching.py}

\subsection{AST Nodes}
\inputminted{python}{../nodes.py}

\subsection{ANLTR Generated Code}
\textbf{Please see the project github if you wish to see this, it's too large an unruly to include in the report.}

\end{document}
